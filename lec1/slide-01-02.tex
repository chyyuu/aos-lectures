\input{../preamble}

%----------------------------------------------------------------------------------------
%	TITLE PAGE
%----------------------------------------------------------------------------------------

\title[第1讲]{第1讲 :Advanced OS Overview} % The short title appears at the bottom of every slide, the full title is only on the title page
\subtitle{第二节:Course Scheduling}

%\author{陈渝} % Your name
%\institute[清华大学] % Your institution as it will appear on the bottom of every slide, may be shorthand to save space
%{
%	清华大学计算机系 \\ % Your institution for the title page
%	\medskip
%	\textit{yuchen@tsinghua.edu.cn} % Your email address
%}
% \date{\today} % Date, can be changed to a custom date
\date{}

\begin{document}

\begin{frame}
\titlepage % Print the title page as the first slide
\end{frame}

%\begin{frame}
%\frametitle{提纲} % Table of contents slide, comment this block out to remove it
%\tableofcontents % Throughout your presentation, if you choose to use \section{} and \subsection{} commands, these will automatically be printed on this slide as an overview of your presentation
%\end{frame}
%
%%----------------------------------------------------------------------------------------
%%	PRESENTATION SLIDES
%%----------------------------------------------------------------------------------------
%
%%------------------------------------------------
%\section{第一节:课程概述} % Sections can be created in order to organize your presentation into discrete blocks, all sections and subsections are automatically printed in the table of contents as an overview of the talk
%%------------------------------------------------
%-------------------------------------------------
\begin{frame}[t]
	\frametitle{Scheduling}

\large
Scheduling
\begin{itemize}
	
	\item Monday (8:00-9:35) 
	\item 4-203, Fourth Building

\end{itemize}
\begin{table}[]
	\begin{tabular}{|c|l|}
		\hline
		1 & Lec 1 Advanced OS Overview                        \\ \hline
		2 & Lec 2 OS Architecture                             \\ \hline
		3 & Lec3+4  System Virtualization Overview            \\ \hline
		4 & Lec5+6  OS/System API/Interface                   \\ \hline
		5 & Lec7+8 OS Kernel and HLL                          \\ \hline
		6 & Lec 9+10 OS for MultiCore Architecture            \\ \hline
		7 & Lec 11+12 OS/System Security                      \\ \hline
		8 & Lec 13+14  Correctness: OS/System Verification    \\ \hline
		9 & Lec 15+16 Invited Talks From Visitors \& Students \\ \hline
	\end{tabular}
\end{table}

\end{frame}

%----------------------------------------------
\begin{frame}[plain]	
	\frametitle{Reading Grading}
	
	\begin{itemize}\Large
		\item 4 Papers Summaries and critiques 60\%
		\begin{itemize}\large
			\item In 12 weeks
			\item Include report docs(4+ or 4000 words in A4 pages) 
			\item Include slides (15+	pages)
			\item Research Areas (14 research domains)
			\begin{itemize}	\large
				\item OS/VMM Architectures
				\item Performance/Multicore
				\item Security/Find Bug
				\item Correctness/Verification
				\item system related......
			\end{itemize}		
			\item The best papers from famous system-related conferences/journals
			\item \href{https://github.com/chyyuu/aos\_course/blob/master/readinglist.md}{ReadingList of OS}			
		\end{itemize}
	\end{itemize}
	
	
\end{frame}
%-------------------------------------------------

%----------------------------------------------
\begin{frame}[plain]	
	\frametitle{Reading Critiques}
	
	\begin{itemize}\Large
		\item Need to address the following:
		\begin{itemize}[<+->]\Large
			
			\item Summary of major innovations
			\item What the problems the paper mentioned?
			\item How about the important related works/papers?
			\item What are some intriguing aspects of the paper?
			\item How to test/compare/analyze the results?
			\item How can the research be improved?
			\item If you write this paper, then how would you do?
			\item What's your test results about the paper?
			\item Give the survey paper list in the same research area
			
		\end{itemize}
	\end{itemize}
	
	
\end{frame}

%----------------------------------------------
\begin{frame}[plain]	
	\frametitle{Project Grading}
	
	\begin{itemize}\Large
		\item One Project		40\%
		\begin{itemize}\large
			\item Didn't finish rcore/ucore/xv6/jos labs
			\begin{itemize}\large
				\item Analyze/Testing/Improving ucore/rcore OS
				\item https://github.com/chyyuu/os\_tutorial\_lab
				\item https://github.com/chyyuu/ucore\_os\_lab
				
			\end{itemize}\pause
			\item Already finished ucore/xv6/jos labs
			\begin{itemize}\large
				\item \href{https://github.com/chyyuu/aos_course/blob/master/readinglist.md}{ReadingList of OS}
				\item find a interesting research topic/project
				\item deep analysis/practice
			\end{itemize}
		\end{itemize}
	\end{itemize}
	
	
\end{frame}

%----------------------------------------------
\begin{frame}[plain]	
	\frametitle{Project Proposal}
	
	\begin{itemize}\Large
		\item Due on the $ 2^{nd} $ week
		\item 2-page written proposal
		\begin{itemize}\large
			\item team members (<=3)
			\item Motivation
			\item The state-of-the-art status
			\item Methodology
			\item Expected results
			\item Timeline
			\item Division of labor among teams
			\item Some references
			
		\end{itemize}
	\end{itemize}
	
	
\end{frame}

%----------------------------------------------
\begin{frame}[plain]	
	\frametitle{Project Midterm/Final Exam}
	
	\begin{itemize}\Large
		\item 10-15 minutes Presentation
		\item Midterm
		\begin{itemize}\large
			\item 3-10 pages paper/report
			\item demo
			
		\end{itemize}
		\item Final
		\begin{itemize}\large
			\item 5-15 pages paper/ project report
			\item demo
			
		\end{itemize}	
	\end{itemize}
	
	
\end{frame}

%----------------------------------------------
\begin{frame}[plain]	
	\frametitle{Other Choose}
	
	\begin{itemize}\Large
		\item I can not understand the paper OR my coding
		ability is poor
		\begin{itemize}\large
			\item Need to talk with me ASAP
			
		\end{itemize}
	\end{itemize}
	
	
\end{frame}
%----------------------------------------------
\end{document}